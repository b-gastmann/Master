% !TEX root = BIGALL-Master.tex
\section{Analytische Methoden}

\subsection{Absorptionsspektroskopie}
	Die Messung der Absorptionsspektren erfolgte mithilfe eines Cary 5000 UV-Vis-NIR mit
	der Software Cary WinUV der Firma Agilent Technologies. 
	Dabei erfolgte bei 350 nm ein automatischer Lampenwechsel und bei 800 nm ein automatischer Detektorwechsel.
	Die	Messungen erfolgten in 3mL Präzisions-Küvetten aus Quarzglas (Suprasil \textregistered) von Hellma	Analytics im Zweistrahlmodus. 
	Dafür wurde zunächst eine Baseline mit reinem Lösemittel gemessen, und anschließend einige \si{\micro\liter} der Nanopartikeldispersionen auf 3mL verdünnt.

\subsection{Röntgenpulverdiffraktometrie (XRD)}
	\todo{Gerät}
	
\subsection{Transmissionselektronenmikroskopie (TEM)}
	Die Messungen erfolgten an einem Tecnai G2 F20 TMP der Firma FEI mit einer 200 kV
	Feldemissions-Kathode. 
	Die Probe wurde auf ein mit Kohlenstoff beschichtetes Kupfergitter der Firma
	Quantifoil getropft und an Luft getrocknet.
	%Die Auswertung erfolgte mithilfe des Programms ImageJ


\subsection{Atomabsorptionsspektroskopie (AAS)}
	Die Bestimmung der Gold-Konzentrationen der Nanopartikellösungen erfolgte mithilfe
	der Atomabsorptionsspektroskopie (AAS). 
	Die Proben wurden mithilfe von 1mL Königswasser über Nacht aufgeschlossen und am
	nächsten Tag in einem Messkolben mit destilliertem Wasser aufgefüllt. Zur Kalibration wurden sechs Lösungen im Konzentrationsbereich zwischen 0 und \SI{2,5}{\milli\gram\liter\tothe{-1}} aus einer
	\SI{1000}{\milli\gram\liter\tothe{-1}} Stammlösung angesetzt.
	Die Messungen erfolgten an einem Flammen-Atomabsorptionsspektrometer AA140 der
	Firma Varian und der Software SpectrAA.