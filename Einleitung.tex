% !TEX root = BIGALL-Master.tex
\section{Einleitung}
Nanopartikel sind für viele technische Anwendungen, wie z.B. in Photovoltaikanlagen, Photodetektoren und als Katalysatoren aufgrund ihrer einzigartigen optoelektronischen Eigenschaften und hohen Oberflächen-zu-Massen-Verhältnissen interessant. 
\autocite{Afzaal2006,Cai2006,Law2005,Yan2003}
Ein Problem dieser Nanopartikel ist es, dass für diese Anwendungen meist keine kolloidalen Lösungen benötigt werden, sondern Nanopartikelanordnungen im festen Zustand.
Aus diesem Grund wurden Nanopartikel zu Gelen formiert, da diese immernoch eine sehr hohes Oberflächen-zu-Massen-Verhältnis besitzen, dabei aber als fester Körper vorliegen.

Synthesen und Eigenschaften von Gelen aus Halbleiter- oder Metallnanopartikeln sind inzwischen sehr gut erforscht.\todo{Quellen}
Ebenso sind inzwischen auch Nanopartikel mit Heterostrukturen aus Metall und Halbleiter gut untersucht.
Was bisher allerdings in der Forschung vernachlässigt wurde, sind Gele mit Heterostrukturen aus Halbleiter- und Metallnanopartikeln.
Und diese sind bisher entweder Gele die aus heterostrukturierten Nanopartikeln gebaut werden,\autocite{Nahar2015,Lesnyak2011} oder Halbleitergele an die Metallnanopartikel gesetzt werden. \autocite{Gill2009,Gill2011}

In dieser Arbeit soll eine Methode entwickelt werden Metallnanopartikelgele mit Halbleiternanopartikel zu besetzen.
Da Gele sehr poröse Stoffe sind und auch in diesen Poren Halbleiternanopartikel herangewachsen werden sollen, muss ein Verfahren für die Synthese von Halbleitern verwendet werden bei denen die Edukte genug Zeit haben um in diese Poren einzudringen.
Gleichzeitig sind nicht zu hohe Temperaturen von Vorteil um die Stabilität der Gele nicht zu gefährden.
Aus diesem Grund werden in dieser Arbeit die Verwendung von Single-Source-Precursoren als Halbleiterquelle untersucht.
Diese bieten sich dadurch an, dass es zu keiner spontanen Reaktion der Edukte kommen kann, da nur die Single-Source-Precursor Lösung zugegeben wird, die ohne Temperaturzufuhr nicht reagiert, was dazu führt, dass der Lösung genug Zeit gelassen werden kann, um in die Poren zu gelangen. 
Gleichzeitig findet die thermische Zersetzung bei unter \SI{300}{\degreeCelsius} statt, wodurch zum einen noch in klassischen Lösemitteln wie TOP gearbeitet werden kann und zum anderen die Stabilität der Gele nicht zu stark angegriffen wird.