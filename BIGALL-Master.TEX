\documentclass[a4paper,
11pt,
BCOR=7mm,
onecolumn,
headsepline,
captions=tableheading,
parskip=half,
ngerman,
bibtotocnumbered,
listof=totoc]{scrartcl}
%Allgemein:
\usepackage[T1]{fontenc}
\usepackage[utf8]{inputenc}
\usepackage[ngerman]{babel}
\usepackage[autooneside=false,automark]{scrlayer-scrpage}
%Schriftart:
\usepackage{lmodern}
%mal schauen:
\usepackage[font=small,labelfont=it]{caption}
\usepackage{here}

\usepackage[onehalfspacing]{setspace} %Für richtigen Abstand der Absätze
%Für Chemie:
\usepackage{chemmacros}
\usepackage{chemformula}
\usepackage{ghsystem}
\usepackage{chemfig}
\usepackage{siunitx}
%Für Tabellen
\usepackage{tabularx}
\usepackage{multicol}
\usepackage{longtable}
\usepackage{booktabs} 	%\*rule-Befehle
\usepackage{array}		%p und b Spalten
%Grafiken
\usepackage{graphicx}
\usepackage{subfig}
%Literatur/Zitieren
\usepackage[autostyle, german=quotes]{csquotes}
\usepackage[sorting=none,doi=false,url=false,autocite=superscript,sortcites=true, labelnumber=true]{biblatex}
%Mehr Zeichen:
\usepackage{amsmath}
\usepackage{amssymb}
%Fußnoten:
\usepackage{savefnmark}
%Aufzählungen
\usepackage{mdwlist}
\usepackage{paralist}
\usepackage{enumerate}
%Für Verweise:
\usepackage{varioref}
\usepackage{hyperref}
\usepackage{cleveref}
%
\usepackage{microtype}
\usepackage{todonotes}

%Literaturdatei:
\bibliography{Literatur}

%Chemfig:
\renewcommand*\printatom[1]{\ensuremath{\mathsf{#1}}}
%SI soll , statt . als Trennzeichen
\sisetup{
output-decimal-marker = {,},
per-mode = symbol,
}
%Absatzabstand
\newcommand{\sref}[1]{(siehe \cref{#1})}

%Allgemeine Einstellungen:
\pagestyle{scrheadings}

%Silbentrennung:
\hyphenation{}

\begin{document}
%% !TEX root = Bigall.tex

\begin{titlepage}
\centering
\begin{figure}
\includegraphics[width=0.3\linewidth,trim=0 0 0 5]{Bilder/LUH}
\hfill
\includegraphics[width=0.3\linewidth,trim=0 0 0 5]{Bilder/LUH}
\end{figure}

\vspace{1cm}
%{\scshape\LARGE Leibniz Universität Hannover \par}
%\vspace{0.5cm}
%{\scshape\Large Masterarbeit\par}
%\vspace{1.5cm}
\noindent\hrule
{\huge\bfseries Synthese und Charakterisierung von nanopartikelbasierten Netzwerkstrukturen \par}
\noindent\hrule
\vspace{1cm}
{\scshape\Large Masterarbeit\par}
am Institut für Physikalische Chemie und Elektrochemie\\
der Leibniz Universität Hannover\\
Laboratorium für Nano- und Quantenengineering\\
\vspace{0.5cm}
im Studiengang Material- und Nanochemie\\
\vspace{0.5cm}
Zur Erlangung des akademischen Grades\\
Master of Science

\vfill
\begin{flushleft}
\begin{tabbing}
Autor: \qquad\qquad\qquad \= Björn Gastmann\\
Matrikelnummer: \> 10004554\\
Erstprüferin: \> Prof. Dr. Nadja Bigall\\
Zweitprüfer: \> Prof. Dr. Peter Behrens\\
Abgabe: \> \today\\
\end{tabbing}
\end{flushleft}
\end{titlepage}
% !TEX root = Bigall.tex

\begin{titlepage}
\centering
\begin{figure}
\includegraphics[width=0.3\linewidth,trim=0 0 0 5]{Bilder/LUH}
\hfill
\includegraphics[width=0.3\linewidth,trim=0 0 0 5]{Bilder/LUH}
\end{figure}

\vspace{1cm}
%{\scshape\LARGE Leibniz Universität Hannover \par}
%\vspace{0.5cm}
%{\scshape\Large Masterarbeit\par}
%\vspace{1.5cm}
\noindent\hrule
{\huge\bfseries Synthese und Charakterisierung von nanopartikelbasierten Netzwerkstrukturen \par}
\noindent\hrule
\vspace{1cm}
{\scshape\Large Masterarbeit\par}
am Institut für Physikalische Chemie und Elektrochemie\\
der Leibniz Universität Hannover\\
Laboratorium für Nano- und Quantenengineering\\
\vspace{0.5cm}
im Studiengang Material- und Nanochemie\\
\vspace{0.5cm}
Zur Erlangung des akademischen Grades\\
Master of Science

\vfill
\begin{flushleft}
\begin{tabbing}
Autor: \qquad\qquad\qquad \= Björn Gastmann\\
Matrikelnummer: \> 10004554\\
Erstprüferin: \> Prof. Dr. Nadja Bigall\\
Zweitprüfer: \> Prof. Dr. Peter Behrens\\
Abgabe: \> \today\\
\end{tabbing}
\end{flushleft}
\end{titlepage}
%
\tableofcontents
\pagebreak

% !TEX root = Bigall.tex
\section{Einleitung}
\pagebreak
% !TEX root = BIGALL-Master.tex
\section{Theoretische Grundlagen}
\subsection{Synthese von Nanopartikeln}
	Bei der Herstellung von Nanopartikeln wird zwischen der \glqq bottom-up\grqq - und der \glqq top-down\grqq - Methode unterschieden.
	Bei der \glqq top-down\grqq - Methode wird mit einem Festkörper angefangen, der dann durch verschiedene physikalische Zerkleinerungsmethoden auf eine gewünschte Partikelgröße gebrochen wird.
	Da diese Brüche nicht gleichmäßig stattfinden, eigenen sich diese Methoden nicht um monodisperse Nanokristalle herzustellen.
	Bei der \glqq bottom-up\grqq - Methode wird ein Nanopartikel aus vielen Monomeren zusammengesetzt, bis zur gewünschten Größe.
	Diese Methode kann mit einzelnen Klemmbausteinen verglichen werden, die zu größeren Gebilden zusammengesetzt werden können.
	Da man bei diesem Verfahren viele Parameter hat, die verändert werden können, durch die die Partikeleigenschaften eingestellt werden können, bietet sich das \glqq bottom-up\grqq - Verfahren zur Synthese monodisperser Partikel besser an.
	Parameter die eingestellt werden können sind u.a. Reaktionstemperatur und Dauer, Konzentration der Edukte, Druck oder Lösungsmittel.
	
	Die theoretische Beschreibung der Bildung monodisperser Nanopartikel geht auf Untersuchungen von LaMer und Dinegar zurück.\autocite{Lamer1950}
	Sie zeigten, dass die Bildung
	monodisperser Kolloide eine zeitlich diskrete (nicht kontinuierliche) Keimbildung
	erfordert, gefolgt von einem langsameren kontrollierten Wachstum der existierenden
	Kerne.
	Bei dem Konzept der sog. „schlagartigen Keimbildung“ muss demnach die
	Keimbildung zu einem einzigen Zeitpunkt ausgelöst werden. Weitere
	Keimbildungsereignisse sind auszuschließen. Startend in einer homogenen Phase setzt
	bei Überwindung der Energiebarriere die Keimbildung ein und es resultiert eine
	heterogene Phase unter homogener Nukleation. \autocite{Park2007}
	
	\begin{figure}[H]
		\centering
		\includegraphics[width=0.6\textwidth]{Bilder/LaMer} 	
		\caption{LaMer-Diagramm: Monomerkonzentration als Funktion der Zeit nach.\autocite{Lamer1950}}
		\label{fig:LaMer}
	\end{figure}

	Im LaMer-Modell existieren 3 Stufen. 
	In Stufe I ist eine kontinuierliche Zunahme der Monomerkonzentration dargestellt.
	Durch die Energiebarriere kommt es auch nach Überschreiten der Sättigungs-Konzentration $C_{Sätt}$ zu keiner Keimbildung.
	Zum Erreichen der Keimbildung in Stufe II muss eine kritische Konzentration $c_{krit}$ überschritten werden.
	Hier werden schlagartig Keime gebildet mit einem kritischen Radius $r_{krit}$, wodurch die Monomerkonzentration wieder unter $c_{krit}$ sinkt, wo dann keine weiteren Keime mehr gebildet werden können.
	Hier wird dann Stufe III erreicht, bei der die Monomere nicht mehr zur Keimbildung, sondern zum Wachstum der Keime genutzt werden, bis ein Gleichgewichtszustand bei der Sättigungskonzentration erreicht wird.
	
%----------------------------------------------
	
	\subsection{Halbleiternanopartikel \& Größenquantisierungseffekt}
	Während Halbleiter als Festkörper eine feste Bandlücke zwischen Valenz- und Leitungsband aufweisen, ist sie bei Nanopartikeln eine von der Partikelgröße abhängige Eigenschaft.
	Allgemein gilt, dass bei kleineren Partikeln die Bandlücke zunimmt.
	\begin{figure}[h]
		\centering
		\includegraphics[width=0.6\linewidth]{Bilder/Quantumsizeeffect}
		\caption{Schematische Darstellung der Veränderung der Bandlücke bei abnehmender Kristallgröße.}
		\label{fig:LCAO}
	\end{figure}
	
    Dies lässt sich über 2 Theorien beschreiben. Zum einen auf der Basis des LCAO-Modells (Linear Combination of Atomic Orbitals) bei dem Nanopartikel als große Moleküle gesehen werden, zum anderen auf Basis der Festkörpertheorie, die Nanopartikel als kleine Festkörper beschreibt.
	
	Bei der LCAO-Methode bilden n Atomorbitale, die gleiche Symmetrie und ähnliche Energie besitzen, durch lineare Kombination, n Molekülorbitale. Wird n sehr groß, kommt es zu einer großen Anzahl Energieniveaus, die sehr nahe beieinander liegen, wodurch kontinuierliche Energiebänder entstehen. Bei Isolatoren und Halbleitern gibt es eine Bandlücke zwischen den Bändern, die bei großen Festkörpern, bei denen die Annahme $n \rightarrow \infty$  gilt, zu einer festen Materialeigenschaft führt. Bei Nanopartikeln gilt diese Näherung allerdings nicht mehr, wodurch sich die kontinuierlichen Bänder mit sinkender Anzahl Atomen wieder in diskrete Energieniveaus aufteilen. Dabei vergrößert sich auch die Bandlücke wieder, wie in \cref{fig:LCAO} dargestellt.
	
	Die Festkörpertheorie: Kommt es zur Absorption eines Photons, kommt es zu einem Loch $h^{+}$  im Valenzband und zu einem Elektron $e^{-}$ im Leiterband, die jedoch aufgrund ihrer Polarität, sich nicht frei im Kristall bewegen können und so ein Paar, das Exziton genannt wird, bilden. Der durchschnittliche Abstand von $e^{-}$ und $h^{+}$ wird Exziton-Bohr-Radius genannt. In einem Nanopartikel ist die Bewegung durch die Partikelgrenzen beschränkt, die als Potentialgrenzen wirken, ähnlich des „Teilchen im Kasten“-Modells, bei dem das Potential außerhalb unendlich ist.
	Ist die Partikelgröße kleiner als der Exziton-Bohr-Radius, muss somit die Bandlücke steigen.
	Da ein kristalliner Festkörper kein konstantes Potential aufweist, sondern eher ein periodisch oszillierendes, wird die Effektive-Masse-Näherung angewandt, bei der sowohl $e^{-}$ als auch $h^{+}$ eine effektive Masse zugeordnet wird, die als Maß für Mobilität der Ladungsträger angesehen werden kann. Daraus ergibt sich die BRUS-Formel:
	\begin{equation}
	\label{eq:BRUS}
	E_{NC}=E_{g}+\frac{h^{2}}{8R^{2}}\left( \frac{1}{m^{*}_{e}}+\frac{1}{m^{*}_{h}}\right)-\frac{1,8e^{2}}{4\pi\epsilon_{0}\epsilon_{r}R} 
	\end{equation}
	\begin{multicols}{2}
		\begin{flushleft}
			$E_{NC}$:	Bandlücke des Nanokristalls\\
			$E_{g}$:	Bandlücke des Festkörpers\\
			$h$:		Plank´sches Wirkungsquantum\\
			$R$:		Partikelgröße\\
			$m^{*}_{e}$: Effektive Masse des $e^{-}$\\
			$m^{*}_{h}$: Effektive Masse des $h^{+}$\\
			$\epsilon_{0}$: Permittivität des Vakuums\\
			$\epsilon_{r}$: relative Permittivität\\
			$e$: Elementarladung\\
		\end{flushleft}
	\end{multicols}
	
%-----------------------------------------------------

    \subsection{Metallnanopartikel \& Oberflächenplasmonen}
    
    Metallische Nanopartikel, die kolloidal in Lösung vorliegen, zeigen ein Absorptionsspektrum.
    Dieses ist aber anders als bei Halbleitern nicht auf Elektronenübergänge zwischen quantisierten Energiezuständen einzelner Elektronen, sondern es kommt zu einer kollektiven Anregung eines Elektronenensambles.
    Diese Anregung wird als Oberflächenplasmonenresonanz bezeichnet. \autocite{Mulvaney1996}
    
    Das Phänomen der lokalisierten Oberflächenresonanz (LOPR) entsteht bei der Wechselwirkung von einer elektromagnetischer Welle und frei beweglichen Elektronen eines Metallnanopartikels.\autocite{Hu2006}
    Das periodische elektromagnetische Feld verursacht dabei eine kollektive Oszillation der leitenden Elektronen, wenn die passende Resonanzfrequenz getroffen wird.
    Da die absorbierten Photonen in Phonen des Metallgitters umgewandelt wird, kann kein Emissionsspektrum gemessen werden.
    Für Nanopartikel mit einem Radius $r<<\lambda$, mit $\lambda$ als Wellenlänge, kann in guter Näherung ein, das Partikel durchdringende Feld, als homogen über das gesamte Partikel angenommen werden. \autocite{Xu1999}
    Dadurch kann vereinfacht angenommen werden, dass nur die Dipolanregung zur Extinktion beiträgt und die Dipolaufspaltung als quasistatisch angenommen werden kann.
    
    \begin{figure}
        \centering
        \includegraphics[width=0.6\linewidth]{Bilder/Muster}
        \caption{Schematische Darstellung der lokalisierten Oberflächenresonanz (LOPR)}
        \label{fig:LOPR}
    \end{figure}
    
    Die plasmonische Resonanzfrequenz ist dabei durch die Permittivität des Metalls, die Größe und Form des Partikels und die Permittivität des umgebenden Mediums. \autocite{Kelly2003,Mock2003}
    Die LOPR-Bande dieser Partikel können sich vom ultravioletten bis in den infraroten Bereich erstrecken. \autocite{Haes2004}
    
    \subsection{Metall-Halbleiter-Nanopartikel}
    Wenn Metallpartikel in Nähe von fluoriszierenden Halbleiter-Nanopartikeln liegen haben diese Einfluss auf die Fluoriszenz dieser.
    Die plasmonischen Metallnanopartikel können als optische Antennen aufgefasst werden, sie können also die zur Anregung der LOPR einfallenden Strahlung im Nahfeld lokalisieren. 
    Je nach Abstand zwischen beiden kann es sowohl zu Fluoriszenzverstärkung als auch zu Fluoriszenzauslöschung kommen. \autocite{Kulakovich2002,Viste2010}
    
    Die Verstärkung kann hauptsächlich auf den Anstieg der Anregungsrate im Fluorophor zurückgeführt werden.\autocite{Chen2008}
    Eine Verstärkung kann durch die chemische Kupplung mit definierten Abstand von plasmonischen Partikeln an die Oberfläche der Halbleiter-Nanopartikel.\autocite{Lee2004}
    
    Die Auslöschung wird durch sehr geringe Abstände oder eine direkte chemische Bindung verursacht.\autocite{Costi2010}
    Hier kommt es an der Grenzfläche zur Ladungstrennung, bei dem ein Elektron aus dem Leitungsband des Halbleiters in die Metalldomäne übertragen wird.
    Dadurch erhöht sich die Menge an strahlungsfreier Relaxation.
    
    
    \subsection{Nanopartikelbasierte Gele}
    
    Allgemein sind Gele Systeme aus mindestens zwei Komponenten, wobei eines ein festes dreidimensionales Netzwerk bildet und von einer Flüssigkeit oder einem Gas ausgefüllt ist. \autocite{Aleman2007}
    
    Der klassische Bildungsmechanismus folgt meist dem Sol-Gel-Verfahren. \autocite{Ziegler2017}
    Dabei wird aus einem flüssigen System über stufenweise Umwandlungen von Vorstufen ein Sol gebildet woraus anschließend ein Gel entsteht. \autocite{Aleman2007}
    Diese Umwandlungen sind meist Polymerisationsreaktionen oder bei anorganischen, oxidschen Gelen Kondensationsreaktionen. 
    Da bei diesen Reaktionen das Material, das das Sol und später das Netzwerk bildet, das gleiche ist, wie das verknüpfende, können diese Prozesse nur für spezielle Materialien durchgeführt werden.
    
    Da viele Eigenschaften von Nanopartikeln form- und größenabhängig sind, diese aber bei diesem klassischen Ansatz nur schwer bis gar nicht einstellbar sind, eignet er sich für die Gelierung von Nanopartikeln nicht gut.
    
    Dieses Problem konnte erstmals in der Arbeit von Brock und Monahan 2004 umgangen werden, durch kontrollierte Destabilisierung der Nanopartikel. \autocite{Mohanan2004,Mohanan2005}
    Diese Destabilisierung erfolgt durch Zugabe eines Oxidationsmittels in eine Lösung von ligandenstabilisierten Nanopartikeln.
    Durch diese Zugabe reagieren Teile der Liganden miteinander, wodurch es zu offenen Stellen an der Oberfläche der Nanopartikel kommt, an denen die Partikel dann aggregieren können.
    Neben diesem Ansatz sind auch andere Gelierungsstrategien bekannt, die in \cref{fig:Destabilisierung} gezeigt sind.
    
    \begin{figure}[H]
        \centering
        \includegraphics[width=0.6\linewidth]{Bilder/Gelierung.png}
        \caption{Strategien für die kontrollierte Destabilisierung von vorab gebildeten kolloidalen Lösungen.\autocite{Aleman2007}}
        \label{fig:Destabilisierung}
    \end{figure}
    
    Auf diese Weise können auch metallische Nanopartikel zu Gelen weiterverarbeitet werden. \autocite{Bigall2009}
    Diese zeigen vielversprechende Eigenschaften bei verschiedenen elektrokatalytischen Prozessen wie Alkoholoxidation, Sauerstoffreduktionsreaktion und Sauerstoffentwicklungsreaktion. \autocite{Cai2018,Shi2018,Zhu2016,Shi2017,Wang2019}
    Edelmetall-Nanostrukturen weisen einzigartige optische Eigenschaften auf, die eine Kopplung zwischen den kollektiven Oberflächenelektronenoszillationen und dem einfallenden elektromagnetischen Feld ermöglichen und so ein dramatisch verstärktes lokales elektrisches Feld für eine oberflächenverstärkte Raman-Streuung ergeben. \autocite{Linic2015,Gao2016}

    
    \subsection{Metall-Halbleiter-Gele}
    Die meisten Untersuchungen von Gelen, die Halbleiter und Metallnanopartikel enthalten, sind Gele mit gemischter Zusammensetzung, bei der Halbleitergele mit Metallnanopartikel wie Gold oder Silber modifiziert werden.\autocite{Nahar2015,Lesnyak2011} 
    Hier können Emissionen beobachtet werden mit Zerfallsraten,  die beim reinen Halbleitergel nicht beobachtet werden können, was auf die Erzeugung alternativer Strahlungszerfälle hinweist. \autocite{Nahar2015}
    Gleichzeitig können bei höheren Raten an Metallpartikeln die Fluoriszenz ausgelöscht werden. \autocite{Lesnyak2011}
\pagebreak
% !TEX root = Bigall.tex
\section{Analytische Methoden}


\pagebreak
% !TEX root = BIGALL-Master.tex
\section{Experimentelles}
	\subsection{Chemikalien}
		\todo{Hersteller}
		\todo{Deutscher Namen?}
		\begin{table}[H]
			\centering
			\caption{Liste aller Verwendeten Chemikalien}
			\label{tab:Chemikalien}
			\begin{tabular}{ll}
				\toprule
				Chemikalie & Hersteller \\
				\midrule
				Aceton&\\
				Methanol (99\%)&\\
				Ethanol&\\
				trockenes Ethanol&\\
				Chloroform&\\
				Hydrogen tetrachloroaurate(III) ($\geq$99,9\%) & \\ 
				borane tert-butylamine(BBA, 97\%) &\\
				Tetralin (99\%)&\\
				Oleylamine ()&\\
				3-Mercaptopropionic acid (MPA,)&\\
				Kaliumhydroxid (KOH,99,9\%)&\\
				Silbernitrat&\\
				Natriumcitrat&\\
				Natriumborohydrid&\\
				Yttrium(III)chlorid Hexahydrat&\\
				Ytterbium(III)chlorid Hexahydrat&\\
				\bottomrule
			\end{tabular}
		\end{table}
	
	\subsection{Synthese von Kupfersulfid durch Single-Source-Precurser}
	\todo{Synthese für Cd und Zn auch aufschreiben oder nur später als gegeben erwähnen?}
	Für die Synthese von Kupfersulfid wird eine thermische Zersetzung eines Kupferkomplexes genutzt, der als Single-Source-Precurser dient. \autocite{JenLaPlante2010}
	\subsubsection{Synthese des Single-Source-Precursors \ch{Cu[DDTC]2}}
	Zur Herstellung des gewünschten Precursors (\ch{Cu[DDTC]2}) werden zwei Lösungen vorbereitet: 
	Einmal werden \SI{0,2218}{\gram} \ch{CuCl2} in \SI{20}{\milli\liter} Wasser gelöst.
	Für die zweite Lösung werden \SI{0,7323}{\gram} \ch{Na[DDTC]* 3 H2O} in \SI{20}{\milli\liter} Wasser gelöst.
	Anschließend werden beide Lösungen zusammengegeben, wobei sich direkt ein dunkler Niederschlag bildet.
	Dieser wird abzentrifugiert und an Luft getrocknet.
	Daraufhin wird es in heißem Chloroform rekristallisiert und erneut an Luft getrocknet.
	Das entstandene \ch{Cu[DDTC]2} wird abgewogen und in TOP aufgenommen, sodass eine 0,11 molare Lösung erhalten wird.
	
	\subsubsection{Herstellung Metallsulfid}
	Es werden \SI{100}{\micro\liter} der 0,11~M \ch{Cu[DDTC]2}-Lösung in ein \SI{8}{\milli\liter}-Schraubdeckelgläschen gegeben, dass in einem Sandbad auf \SI{290}{\degreeCelsius} unter Stickstoffatmosphäre erhitzt wird.
	Dies wird solange erhitzt, bis das gesamte Lösemittel verdampft ist und ein schwarzer trockener Rückstand erkennbar ist.
	Dieser wird in \SI{0,5}{\milli\liter} Toluol aufgenommen, anschließend \SI{1}{\milli\liter} Methanol dazugegeben und abzentrifugiert.
	Der Rückstand wird in \SI{0,5}{\milli\liter} Toluol aufgenommen.
	
	\subsection{Synthese von CuS mit \ch{Cu[DDTC]2} und \ch{CuCl2}}
	Da bekannt ist, dass Chloridionen einen starken Einfluss auf das Nukleationsverhalten bei Nanopartikeln haben kann\autocite{Hinrichs2016}, wurde ein Teil des Single-Source-Precursors durch \ch{CuCl2} substituiert.
	
	\subsection{Synthese von Kupfersulfid in Anwesenheit von Goldnanopartikeln}
	
	Bei dieser Synthese wurde die CuS-Synthese, wie oben beschrieben, durchgeführt, mit einem Anteil an Gold-Nanopartikeln, mit der Absicht, dass die CuS-Bildung um die Nanopartikel herum stattfindet.	
	Es werden \SI{100}{\micro\liter} der 0,11~M \ch{Cu[DDTC]2}-Lösung und verschiedene Mengen (\cref{tab:Au_Cu_Ratio} der Goldnanopartikel in Hexan in ein \SI{8}{\milli\liter}-Schraubdeckelgläschen zusammengegeben und in einem Sandbad unter Stickstoffatmosphäre auf \SI{290}{\degreeCelsius} erhitzt, bis das Lösemittel komplett verdampft ist.
	Der Rückstand wurde in \SI{0,5}{\milli\liter} Toluol aufgenommen, \SI{1}{\milli\liter} Ethanol dazugegeben, abzentrifugiert und anschließend wieder in \SI{0,5}{\milli\liter} Toluol aufgenommen.
	
	\begin{table}[H]
		\centering
		\caption{Die verschiedenen genutzten Verhältnisse aus Goldnanopartikeln und \ch{Cu[DDTC]2}.}
		\label{tab:Au_Cu_Ratio}
		\begin{tabular}{ll}
			\toprule
			V(Au)/\si{\micro\liter}&V(\ch{Cu[DDTC]2})/\si{\micro\liter}\\
			\midrule
			10&100\\
			20&100\\
			30&100\\
			40&100\\
			50&100\\
			100&100\\
			\bottomrule
		\end{tabular}
	\end{table}

	\subsection{Herstellung von Gelen}
		
		\subsubsection{Herstellung von Hydrogelen durch Yttrium und Ytterbium}
		
		
			\paragraph{Synthese von Gold-Nanopartikel in organischen Lösungsmitteln}
		
			\SI{200}{\milli\gram} Hydrogen tetrachloroaurate(III) werden gelöst in einer Lösung aus \SI{10}{\milli\liter} Tetralin und \SI{10}{\milli\liter} Oleylamine und in einem Eisbad für 20 Minuten gerührt.
			\SI{0,087}{\gram} BBA werden in \SI{1}{\milli\liter} Tetralin und \SI{1}{\milli\liter} Oleylamine gegeben und in einem Ultraschallbad gelöst und anschließend in gekühlte Lösung injiziert.
			Anschließen wird das Reaktionsgemisch für weitere \SI{2}{\hour} im Eisbad gerührt.
			Daraufhin wird die Lösung auf 2 \SI{50}{\milli\liter}-Zentrifugengefäße gleichmäßig aufgeteilt und zum Fällen der Partikel mit Aceton auf \SI{50}{\milli\liter} aufgefüllt und bei 8500~rpm für \SI{10}{\minute} zentrifugiert.
			Nach dem Zentrifugieren wurden die Partikel in \SI{3}{\milli\liter} je Zentrifugengefäß redispergiert.
			%Anschließend wurden die beiden Lösungen wieder zusammen in ein 25ml Schraubdeckelgläschen gegeben. Name: BG-AU-NP1. Verweis auf Liu2015
				
			\paragraph{Phasentransfer der Gold-Nanopartikel}
		
			Für den Phasentransfer werden je \SI{1}{\milli\liter} der Goldlösung in \SI{5}{\milli\liter} einer methanolischen \SI{0,1}{M}~KOH-Lösung gegeben und anschließend \SI{250}{\micro\liter} MPA dazugegeben und für \SI{4}{\hour} geschwenkt.
			Nach dem Abzentrifugieren (8500~rpm, 10 Minuten) wird der Rückstand in 4mL wässriger KOH-Lösung aufgenommen und dann \SI{1}{\milli\liter} Chloroform dazugegeben. 
			Da nach erneutem Zentrifugieren sich kein fester Rückstand gebildet hatte, wurden \SI{10}{\milli\liter} Aceton dazugegeben und das Zentrifugieren wiederholt.
				
			\paragraph{Gelierung der Gold-Nanopartikel mit Yttrium und Ytterbium}
				
			Es werden von den phasentransferierten Gold-Nanopartikeln, nach vorheriger Bestimmung der Goldkonzentration durch AAS-Messungen, Lösungen mit den Konzentrationen 0,625; 1,5 und \SI{2,5}{\gram\per\liter} hergestellt.
			Mit Yttrium(III)chlorid Hexahydrat bzw. Ytterbium(III)chlorid Hexahydrat werden je 3 Lösungen (1; 5; \SI{10}{mM}) hergestellt.
			Es werden immer \SI{200}{\micro\liter} Goldlösung und \SI{25}{\micro\liter} Yttrium- bzw. Ytterbium-Lösung dazugegeben zusammen in ein \SI{2}{\milli\liter}-Zentrifugengefäß gegeben und für \SI{24}{\hour} stehen gelassen.
			Dies wurde für jede Kombination aus Goldlösung und Y/Yb-Lösung, insgesammt 18, durchgeführt.
			Es bildet sich bei allen unten im Gefäß ein kleiner dunkler Klumpen.
			  
		\subsubsection{Herstellung von citratstabilisierten Hydrogelen}
			Hier werden im ersten Schritt Gold- und Silbernanopartikel in wässriger Lösung synthetisiert und aufkonzentriert.
			Im zweiten Schritt wird dann die Gelierung ausgelöst. \autocite{Bigall2009} 
		 
			\paragraph{Synthese Goldnanopartikel in wässriger Phase mit Natriumcitrat}
			
			\SI{29}{\milli\liter} 0,2\%ige Goldchlorid-Lösung werden in \SI{500}{\milli\liter} dest. Wasser gegeben.
			Es werden \SI{11,6}{\milli\liter} einer 1\%igen Natriumcitratlösung dazugegeben und nach 30 Sekunden \SI{5,8}{\milli\liter}	einer eiskalten Lösung aus \SI{0,085}{\gram} Natriumborohydrid und \SI{0,5}{\gram} Natriumcitrat in \SI{50}{\milli\liter} dest. Wasser dazugegeben.
			Anschließend werden die Goldpartikel durch Zenrifugenfilter auf 10mL aufkonzentriert.
			
			\paragraph{Synthese Silbernanopartikel in wässriger Phase mit Natriumcitrat}
			
			\SI{12}{\milli\liter} 0,2\%ige Silbernitrat-Lösung werden in \SI{488}{\milli\liter} dest. Wasser gegeben und auf \SI{100}{\degreeCelsius} erhitzt.
			Anschließend werden \SI{11,6}{\milli\liter} einer 1\%igen Natriumcitratlösung dazugegeben.
			Nach 30 Sekunden werden  \SI{5,5}{\milli\liter}	einer eiskalten Lösung aus \SI{0,038}{\gram} Natriumborohydrid und \SI{0,5}{\gram} Natriumcitrat in \SI{50}{\milli\liter} dest. Wasser dazugegeben.
			Nach 2 Minuten wird alles im Dunkeln abgekühlt.
			Anschließend werden die Silberpartikel durch Zenrifugenfilter auf 10mL aufkonzentriert.
			
			\paragraph{Gelierungsprozess}
			    \begin{description}
			    \item[Gele aus Gold-Nanopartikeln:]
			    \SI{500}{\micro\liter} der Goldlösung werden in ein 2mL-Zentri"-fugen"-gefäß gegeben und anschließend \SI{40}{\micro\liter} einer 1\%igen Wasserstoffperoxid-Lösung gegeben und anschließend für 30 Tage dunkel gelagert.
			    \item[Gele aus Silber-Nanopartikeln:]
			    \SI{500}{\micro\liter} der Silberlösung werden in ein 2mL-Zentri"-fugen"-gefäß gegeben und anschließend \SI{200}{\micro\liter} einer 3\%igen Wasserstoffperoxid-Lösung gegeben und anschließend für 30 Tage dunkel gelagert.
			    \item[Bimetallisches Gel aus Gold/Silber-Nanopartikeln:]
			    \SI{500}{\micro\liter} der Goldlösung und 	\SI{500}{\micro\liter} der Silberlösung werden in ein 2mL-Zentri"-fugen"-gefäß gegeben und anschließend für 30 Tage dunkel gelagert.
			    \end{description}
			   	
			   	In den Gefäßen mit reinem Gold bildeten sich keine Gele und es war weiterhin nur eine rote Lösung zu erkennen.
			   	In den Gefäßen mit reinem Silber bildete sich ein gelb-brauner Schleim am Boden, jedoch kein festes Gel.
			   	Nur in den Gefäßen mit Gold und Silber bildeten sich Gele, die für weitere Versuche verwendet werden konnten.
			
		\subsubsection{Ethanolischer Ansatz für direkte Synthese von Gelen}
			Im Gegensatz zu den vorherigen Ansätzen, sind hier Nanopartikelbildung und Gelierung keine zwei voneinander getrennte Syntheseschritte.
			Hier werden direkt aus der ethanolischen Metallsalzlösung die Gele gebildet. \autocite{Georgi2019}
			
			Unter Schutzatmosphäre werden \SI{39}{\milli\gram} Hydrogen tetrachloroaurate(III) in \SI{8}{\milli\liter} trockenem Ethanol gegeben.
			Eine zweite Lösung wird hergestellt aus \SI{11}{\milli\gram} NaBH$_4$ in \SI{6}{\milli\liter} trockenem Ethanol gegeben.
			Es werden 6 Proben präpariert, indem jeweils \SI{1,33}{\milli\liter} der Goldlösung in ein \SI{8}{\milli\liter}-Schraubdeckelglas gegeben wird und anschließend \SI{1}{\milli\liter} der zweiten Lösung dazugegeben wird.
			Die Lösung färbt sich sofort dunkel.
			Nach einer Stunde hat sich die Lösung klar gefärbt und es hatte sich entweder ein dunkel Bodensatz gebildet oder es schwamm ein Klumpen an der Oberfläche.
			Bei den Proben mit Bodensatz wurden die Gläser leicht schräg gehalten und gedreht, wodurch sich der Bodensatz zusammenklumpte und wie bei den anderen Proben, dieser Klumpen dann an der Oberfläche schwamm.  
			
	\subsection{Phasentransfer der Gele}
	
			Bei den Gelen in wässriger Lösung wurde ein Austausch des Lösungsmittel vorgenommen, um Gele mit TOP als flüssige Phase zu erhalten.
			\subsubsection{Citratstabilisierte Gele}
				Bei den citratstabilisierten Gelen wurde im ersten Schritt das Lösemittel bis auf die Höhe des Gels  entnommen und das 2mL-Zentri"-fugen"-gefäß mit Ethanol aufgefüllt.
				Dieser Schritt wurde insgesamt dreimal durchgeführt und nach 24 Stunden drei weitere Male.
				Bei dem Versuch anschließend gleiches mit TOP zu versuchen, ergab sich das Problem, dass sich 2 getrennte Phasen bildeten, bei der auch nach 24 Stunden TOP, trotz höherer Dichte als Ethanol, die obere Phase bildete.
				Aus diesem Grund wurde ein weiterer Zwischenschritt über Toluol eingebaut, bei dem analog zu Ethanol vorgegangen wurde.
				Diesmal bildete sich bei Zugabe von TOP keine Phasentrennung und es wurde analog zu den vorherigen Schritten vorgegangen.
				Die Gele wurden daraufhin vorsichtig in ein \SI{8}{\milli\liter}-Schraubdeckelglas überführt, indem beide Gefäße komplett mit TOP gefüllt wurden und so das Gel langsam in das Schraubdeckelglas absinken konnte.
			
			\subsubsection{Gele durch Zugabe von Yttrium und Ytterbium}
				Bei diesen Gelen wurde versucht analog zu den citratstabilisierten vorzugehen.
				Bei dem Versuch Ethanol zuzugeben, ergab sich jedoch das Problem, dass das Gel sofort nach der Zugabe von Ethanol in viele Teile zerbrach und sich im ganzen Lösungsmittel verteilte.
				Aus diesem Grund wurde ein langsamer Phasentransfer versucht indem ein Lösemittelgemisch aus 90\% \ch{H2O} und 10\% Ethanol dazugegeben wurde.
				Doch auch hier konnte der gleiche Effekt wie vorher beobachtet werden.
				Auch bei dem Versuch, anstatt das alte Lösemittel zu entnehmen und mit neuem zu ersetzen, das 90:10-Gemisch auf die Oberfläche der Lösung langsam aufzutragen, ergab sich gleiches Problem.
				Aus diesem Grund konnten diese Gele für weitere Versuche nicht verwendet werden.   
	
	\subsection{Synthese von Kupfersulfid in Anwesenheit von Gelen}
		
			Die Parameter der Synthese wurden immer wieder variiert, wobei der Versuchsaufbau immer identisch ist.
			Exemplarisch wird hier eine Synthese beschrieben:
			
			Zu einem Gel, dass durch ethanolischen Ansatz in direkter Synthese hergestellt wurde, in einem \SI{8}{\milli\liter}-Schraubdeckelglas wurden \SI{200}{\micro\liter} der 0.11~M~\ch{Cu[DDTC]2} in TOP Lösung hinzugegeben.
			Das \SI{8}{\milli\liter}-Schraubdeckelglas wird daraufhin in ein Sandbad gegeben.
			Da der Boden des Schraubdeckelglases Flach ist, und deutlicher größer als der Querschnitt der Gele, wird das Glas leicht schräg ins Sandbad gegeben mit dem Gel an der tiefsten Stelle, um eine möglichst große Fläche das Gels für eine möglichst lange Zeit in der Lösung liegen zu haben.
			Anschließend wird eine Schutzatmosphäre aus Stickstoff eingeleitet und alles auf \SI{290}{\degreeCelsius} erhitzt.
			Diese Temperatur wird gehalten bis das gesamte Lösemittel verdampft ist.
			Nach dem Abkühlen werden \SI{2}{\milli\liter} Aceton zugegeben. 
			Nach 24 Stunden wird das Aceton entnommen und \SI{2}{\milli\liter} Aceton zugegeben.
			Dies wird zweimal wiederholt. 
			
			\subsubsection{Variationen der Synthese}
				
				\paragraph{Variation der Temperatur}
				\qquad\newline
					Die Synthesen wurden wie oben durchgeführt, allerdings wurde die Reaktionstemperatur variiert, wie \cref{tab:Temperatur} zeigt.
					
					\begin{table}[htbp]
						\centering
						\caption{Überblick über die verwendeten Reaktionstemperaturen}
						\label{tab:Temperatur}
						\begin{tabular}{cc}
							\toprule
							Probe & Temperatur/\si{\degreeCelsius}\\
							\midrule
							T1 & 290\\
							T2 & 250\\
							T3 & 200\\
							\bottomrule
						\end{tabular}
					\end{table}
				
				\paragraph{Variation des Metalls im Single-Source-Precursor}
				\qquad\newline
					Neben Kupfer als \ch{Cu[DDTC]2}-Komplex wurden auch Zink und Cadmium als \ch{Zn[DDTC]2} bzw. \ch{Cd[DDTC]2}-Komplex verwendet.
					
				\paragraph{Zusatz von Chlorid}
				\qquad\newline
					Zusätzlich zum Metallkomplex wurde hier eine bestimmte Menge des entsprechenden Metallchlorids zugegeben.
					Diese Versuche wurden für Cadmium und Zink durchgeführt.
					Das Metallchlorid wurde dabei in TOP gelöst und auf die gleiche Konzentration wie der Metallkomplex angepasst.
					Da sich die Metallchloride nur sehr schlecht lösten, wurden sie für 30 Minuten in ein Ultraschallbad gelegt.
					Außerdem wurden die Lösungen vor Gebrauch immer für 15 Minuten in ein Ultraschallbad gehalten. 
					Die verwendeten Mengenverhältnisse können \cref{tab:Chlorid} entnommen werden.
					
					\begin{table}[htbp]
						\centering
						\caption{Überblick über die verwendeten Mengen von Komplex und Chlorid-Lösung}
						\label{tab:Chlorid}
						\begin{tabular}{cccc}
							\toprule
							Metall & V(Komplex)/\si{\micro\liter} & V(Chlorid)\si{\micro\liter} & Verhältnis \\ 
							\midrule
							Cd & 160 & 40 & 4:1\\
							Zn & 160 & 40 & 4:1\\
							Cd & 100 & 100 & 1:1\\
							Zn & 100 & 100 & 1:1\\
							Cd & 180 & 18 & 10:1\\
							Zn & 180 & 18 & 10:1\\
							\bottomrule
						\end{tabular}
					\end{table}
				
				\paragraph{Variation der Konzentration}
				\qquad\newline
					Die Konzentration der \ch{Cd[DDTC]2} - Lösung wurde variiert. 
					Dabei wurden sowohl Konzentrationsänderungen durchgeführt, als auch Chloridanteile variiert.
					Die Cadmiumchlorid Konzentration der Lösung wurde dabei der Komplexkonzentration angepasst.  
					
					\begin{table}[htbp]
						\centering
						\caption{Überblick der verwendeten Konzentrationen und Chloridanteile}
						\label{tab:Konz-Chlorid}
						\begin{tabular}{cccc}
							\toprule
							Konzentration/\si{\mol\liter\tothe{-1}}& V(\ch{Cd[DDTC]2})/\si{\micro\liter} & V(\ch{CdCl2})/\si{\micro\liter} & Verhältnis\\
							\midrule
							0,05 & 200 & 0 &-\\
							0,05 & 160 & 40 & 4:1\\
							0,05 & 100 & 100 & 1:1\\
							0,05 & 180 & 20 & 10:1\\
							0,01 & 200 & 0 &-\\
							\bottomrule
						\end{tabular}
					\end{table}
				
			\paragraph{Variation des Gels}
			\qquad\newline
				Neben den ethanolischen Gelen wurde das Gel geändert und die citratstabilisierten bimetallischen Au/Ag-Gele verwendet.
				Hier wurde sich auf Cadmium als Metall beschränkt und verschiedene Chloridmengen verwendet.
				
				\begin{table}[htbp]
					\centering
					\caption{Verwendete Mengenverhältnisse von \ch{Cd[DDTC]2} und \ch{CdCl2} bei Au/Ag-Gelen}
					\label{tab:}
					\begin{tabular}{ccc}
						\toprule
						 V(\ch{Cd[DDTC]2})/\si{\micro\liter} & V(\ch{CdCl2})/\si{\micro\liter} & Verhältnis\\
						\midrule
						200 & 0 & -\\
						160 & 40 & 4:1\\
						100 & 100 & 1:1\\ 
						\bottomrule
					\end{tabular}
				\end{table}
				
				
					
					
			
			
	
\pagebreak
% !TEX root = BIGALL-Master.tex
\section{Messungen und Auswertung}

\subsection{Synthese Kupfersulfidnanopartikel}
	Bei dieser Synthese war das Ziel Kupfersulfid aus dem Single-Source-Precursor \ch{Cu[DDTC]2} als Nanopartikel herzustellen. 
	Die Partikel wurden per UV-vis-NIR-Spektroskopie und mit TEM untersucht.
	
	\begin{figure}[H]
		\centering
		\includegraphics[width=0.6\textwidth]{Bilder/Muster} 	
		\caption{Absorptionsspektrum der CuS-Nanopartikel in Toluol gemessen.}
		\label{fig:UV-CuS}
	\end{figure}

	In dem Absorptionspektrum sind... \todo{UV}
	
	\begin{figure}[H]
		\centering
		\includegraphics[width=0.6\textwidth]{Bilder/MUSTER} 	
		\caption{TEM-Bilder der CuS-Nanopartikel}
		\label{fig:TEM-CuS}
	\end{figure}

	Auf den TEM-Bildern erkennt man, dass die hergestellten Partikel. \todo{TEM}
	
\subsection{Synthese Goldnanopartikel}
	Bei dieser Synthese war es das Ziel möglichst monodisperse Goldnanopartikel herzustellen.
	Die Partikel wurden per UV-vis-Spektroskopie und mit TEM untersucht.
	
	\begin{figure}[H]
		\centering
		\includegraphics[width=0.6\textwidth]{Bilder/Muster} 	
		\caption{Absorptionsspektrum der Goldnanopartikel in Hexan gemessen.}
		\label{fig:UV-AuNP}
	\end{figure}
	
	In dem Absorptionspektrum sind. \todo{UV}
	
	\begin{figure}[H]
		\centering
		\includegraphics[width=0.6\textwidth]{Bilder/MUSTER} 	
		\caption{TEM-Bilder der Gold-Nanopartikel}
		\label{fig:TEM-AuNP}
	\end{figure}
	
	Auf den TEM-Bildern erkennt man, dass die hergestellten Partikel. \todo{TEM}
	
	Die Partikel zeigten also eine zufriedenstellende Größenverteilung
	

	
	
\subsection{Synthese von CuS mit \ch{Cu[DDTC]2} und \ch{CuCl2}}

	Um eine 
	
	\begin{figure}[H]
		\centering
		\includegraphics[width=0.6\textwidth]{Bilder/Muster} 	
		\caption{TEM-Bilder von CuS-Nanopartikel, aus einem Gemisch von \ch{Cu[DDTC]2} und \ch{CuCl2}.}
		\label{fig:TEM-CuCl}
	\end{figure}
	
\subsection{Synthese von Kupfersulfid in Anwesenheit von Goldnanopartikeln}

	Um das Nukleationsverhalten bei der CuS-Synthese zu Untersuchen, wurde diese Reaktion in Anwesenheit von Goldnanopartikeln in verschiedenen Verhältnissen untersucht.
	Es wurden von diesen Mischungen jeweils UV-vis-NIR-Messungen und TEM-Aufnahmen vorgenommen.	
	

\subsection{Synthese der Gele}
	
	\subsubsection{Untersuchung von Hydrogelen durch Yttrium und Ytterbium}
	
	
	
	\subsubsection{Untersuchung von citratstabilisierten Hydrogelen}
	
		Da bei den Versuchen mit reinem Gold und Silber keine festen Gele entstanden sind wurden diese auch nicht weiter untersucht.
		Von den bimetallischen Gelen aus Gold- und Silbernanopartikeln wurden TEM-Bilder aufgenommen.
		
	\subsubsection{Untersuchung der Gele aus ethanolischem Ansatz}
		
		Die Gele zeigen 
 
\pagebreak
% !TEX root = BIGALL-Master.tex
\section{Zusammenfassung}

Ziel der Arbeit war es eine Methode zu entwickeln an vorhandenen Nanopartikelgelen andere Materialien heranzuwachsen.
Die hierfür gewählten Materialien waren metallische Nanopartikelgele an denen Halbleitermaterialien herangewachsen wurde.

Da bei der Arbeit mit Gelen zu hohe Temperaturen zu vermeiden waren und gleichzeitig eine Synthesemethode gewählt werden musste, bei der die Edukte nicht sofort bei Zugabe reagierten, sondern genug Zeit haben in die inneren Poren der Gele vorzudringen wurde als grundlegende Methode hier die thermische Zersetzung von Single-Source-Precursoren verwendet.

Als Single-Source-Precursoren wurden   verwendet, da sich diese schon bei Temperaturen von unter \SI{300}{\degreeCelsius} zu Metallsulfiden zersetzen.
Die Komplexe wurden \ch{Cu[DDTC]2}, \ch{Cd[DDTC]2} und \ch{Zn[DDTC]2} hergestellt und ihre Bildung von Nanopartikeln an Gelen untersucht.

Bei der Synthese der Gele wurden verschiedene bekannte Methoden verwendet.
Es wurde zum einen ein klassischen Verfahren zur Herstellung von Gelen aus citratstabilisierten Nanopartikel durch Ligandenoxidation durch \ch{H2O2} verwendet,
zum anderen ein neuerer Ansatz bei dem die Nanopartikel durch Zugabe von Yttrium und Ytterbium zur Gelierung gebracht werden.
Des weiteren wurde ein neueres Verfahren verwendet, bei dem Nanopartikelbildung und Gelbildung in einem Schritt zusammengefasst sind.
Vor allem Reaktionen mit den zuletzt genannten Gelen wurden in dieser Arbeit untersucht.

Es zeigte sich, dass sich durch die thermische Zersetzung der Single-Source-Precursoren Anlagerungen an den Gelen bildeten.
Je nach verwendetem Precursor variierten diese jedoch von fast kompletter Ummantelung des Gels bei \ch{Cd[DDTC]2} bis zu keiner Anlagerung bei \ch{Zn[DDTC]2}, was durch die Zugabe von Zinkchlorid allerdings verbessert werden konnte.
Es konnte sowohl bei den Gelen aus ethanolischen Ansatz als auch aus den citratstabilisierten Nanopartikeln Anlagerungen beobachtet werden.

Insgesamt konnte hier also mit dem Prinzip der Zersetzung von Single-Source-Precursoren eine Methode entwickelt werden, mit der das nachträgliche Auftragen von Partikeln an Gelen ermöglicht wird.

Weiterführend könnte man Tests mit anderen Metallen in Diethyldithiocarbamat-Komplexen durchführen oder andere Komplexe wie Dimethyldithiocarbamat- oder Hexylmethyldithiocarbamat-Komplexe verwenden.
Des weiteren kann dieser Prozess auch mit anders beschaffenen Gelen verwendet werden.


\chapter*{Danksagung}

Die vorliegende Masterarbeit entstand im Arbeitskreis von Prof. Dr. Nadja Bigall am Institut für Physikalische Chemie und Elektrochemie der Leibniz Universität Hannover. Ich möchte mich an dieser Stelle bei allen bedanken, die mich während dieser Arbeit sowohl fachlich als auch moralisch unterstützt haben. Mein besonderer Dank gilt
dabei vor allem:

\begin{itemize}
	\item \textbf{Frau Prof. Dr. Nadja Bigall} für die Möglichkeit diese Arbeit in ihrem Arbeitskreis zu verfassen und ihre stetige Bereitschaft für produktive und interessante Hinweise.
	\item \textbf{Herrn Prof. Dr. Peter Behrens} dafür, dass er als Zweitprüfer zur Verfüngung stand.
	\item \textbf{Pascal Rusch} für seine stets hilfsbereite Art und die TEM-Messungen.
	\item Dem gesamten Arbeitskreis Bigall für das tolle Arbeitsklima.
	\item Meiner Familie und meinen Freunden die immer für mich da sind.
	
\end{itemize}

\pagebreak

\chapter*{Eidesstattliche Erklärung}

Hiermit versichere ich, die vorliegende Masterarbeit selbstständig, ohne
fremde Hilfe und ohne Benutzung anderer als der von mir angegebenen Quellen
angefertigt zu haben. Alle aus fremden Quellen direkt oder indirekt übernommenen
Gedanken sind als solche gekennzeichnet. Die Arbeit wurde noch
keiner Prüfungsbehörde in gleicher oder ähnlicher Form vorgelegt.

\vspace{5cm}

\noindent Hannover, den 31. März 2020
\vspace{2cm}
\hrule
\vspace{5mm}
\noindent Björn Gastmann
\pagebreak
\printbibliography
\pagebreak
\section{Anhang}
\listoffigures
\pagebreak
\listoftables
%-------------------
\pagebreak
\pagestyle{plain}
\section*{Danksagung}

Die vorliegende Masterarbeit entstand im Arbeitskreis von Prof. Dr. Nadja Bigall am Institut für Physikalische Chemie und Elektrochemie der Leibniz Universität Hannover. Ich möchte mich an dieser Stelle bei allen bedanken, die mich während dieser Arbeit sowohl fachlich als auch moralisch unterstützt haben. Mein besonderer Dank gilt
dabei vor allem:

\begin{itemize}
	\item \textbf{Frau Prof. Dr. Nadja Bigall} für die Möglichkeit diese Arbeit in ihrem Arbeitskreis zu verfassen und ihre stetige Bereitschaft für produktive und interessante Hinweise.
	\item \textbf{Herrn Prof. Dr. Peter Behrens} dafür, dass er als Zweitprüfer zur Verfüngung stand.
	\item \textbf{Pascal Rusch} für seine stets hilfsbereite Art und die TEM-Messungen.
	\item Dem gesamten Arbeitskreis Bigall für das tolle Arbeitsklima.
	\item Meiner Familie und meinen Freunden die immer für mich da sind.
	
\end{itemize}

\pagebreak

\section*{Eidesstattliche Erklärung}

Hiermit versichere ich, die vorliegende Masterarbeit selbstständig, ohne
fremde Hilfe und ohne Benutzung anderer als der von mir angegebenen Quellen
angefertigt zu haben. Alle aus fremden Quellen direkt oder indirekt übernommenen
Gedanken sind als solche gekennzeichnet. Die Arbeit wurde noch
keiner Prüfungsbehörde in gleicher oder ähnlicher Form vorgelegt.

\vspace{5cm}

\noindent Hannover, den 31. März 2020
\vspace{2cm}
\hrule
\vspace{5mm}
\noindent Björn Gastmann
\end{document}