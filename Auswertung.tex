% !TEX root = BIGALL-Master.tex
\section{Messungen und Auswertung}

\subsection{Synthese Kupfersulfidnanopartikel}
	Bei dieser Synthese war das Ziel Kupfersulfid aus dem Single-Source-Precursor \ch{Cu[DDTC]2} als Nanopartikel herzustellen. 
	Die Partikel wurden per UV-vis-NIR-Spektroskopie und mit TEM untersucht.
	
	\begin{figure}[H]
		\centering
		\includegraphics[width=0.6\textwidth]{Bilder/Muster} 	
		\caption{Absorptionsspektrum der CuS-Nanopartikel in Toluol gemessen.}
		\label{fig:UV-CuS}
	\end{figure}

	In dem Absorptionspektrum sind... \todo{UV}
	
	\begin{figure}[H]
		\centering
		\includegraphics[width=0.6\textwidth]{Bilder/MUSTER} 	
		\caption{TEM-Bilder der CuS-Nanopartikel}
		\label{fig:TEM-CuS}
	\end{figure}

	Auf den TEM-Bildern erkennt man, dass die hergestellten Partikel. \todo{TEM}
	
\subsection{Synthese Goldnanopartikel}
	Bei dieser Synthese war es das Ziel möglichst monodisperse Goldnanopartikel herzustellen.
	Die Partikel wurden per UV-vis-Spektroskopie und mit TEM untersucht.
	
	\begin{figure}[H]
		\centering
		\includegraphics[width=0.6\textwidth]{Bilder/Muster} 	
		\caption{Absorptionsspektrum der Goldnanopartikel in Hexan gemessen.}
		\label{fig:UV-AuNP}
	\end{figure}
	
	In dem Absorptionspektrum sind. \todo{UV}
	
	\begin{figure}[H]
		\centering
		\includegraphics[width=0.6\textwidth]{Bilder/MUSTER} 	
		\caption{TEM-Bilder der Gold-Nanopartikel}
		\label{fig:TEM-AuNP}
	\end{figure}
	
	Auf den TEM-Bildern erkennt man, dass die hergestellten Partikel. \todo{TEM}
	
	Die Partikel zeigten also eine zufriedenstellende Größenverteilung
	

	
	
\subsection{Synthese von CuS mit \ch{Cu[DDTC]2} und \ch{CuCl2}}

	Um eine 
	
	\begin{figure}[H]
		\centering
		\includegraphics[width=0.6\textwidth]{Bilder/Muster} 	
		\caption{TEM-Bilder von CuS-Nanopartikel, aus einem Gemisch von \ch{Cu[DDTC]2} und \ch{CuCl2}.}
		\label{fig:TEM-CuCl}
	\end{figure}
	
\subsection{Synthese von Kupfersulfid in Anwesenheit von Goldnanopartikeln}

	Um das Nukleationsverhalten bei der CuS-Synthese zu Untersuchen, wurde diese Reaktion in Anwesenheit von Goldnanopartikeln in verschiedenen Verhältnissen untersucht.
	Es wurden von diesen Mischungen jeweils UV-vis-NIR-Messungen und TEM-Aufnahmen vorgenommen.	
	

\subsection{Synthese der Gele}
	
	\subsubsection{Untersuchung von Hydrogelen durch Yttrium und Ytterbium}
	
	
	
	\subsubsection{Untersuchung von citratstabilisierten Hydrogelen}
	
		Da bei den Versuchen mit reinem Gold und Silber keine festen Gele entstanden sind wurden diese auch nicht weiter untersucht.
		Von den bimetallischen Gelen aus Gold- und Silbernanopartikeln wurden TEM-Bilder aufgenommen.
		
	\subsubsection{Untersuchung der Gele aus ethanolischem Ansatz}
		
		Die Gele zeigen 
 