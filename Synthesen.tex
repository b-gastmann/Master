% !TEX root = Bigall.tex
\section{Experimentelles}
	\subsection{Chemikalien}
		\todo{Hersteller}
		\todo{Deutscher Namen?}
		\begin{table}[H]
			\centering
			\caption{Liste aller Verwendeten Chemikalien}
			\label{tab:Chemikalien}
			\begin{tabular}{ll}
				\toprule
				Chemikalie & Hersteller \\
				\midrule
				Aceton&\\
				Methanol (99\%)&\\
				Ethanol&\\
				trockenes Ethanol&\\
				Chloroform&\\
				Hydrogen tetrachloroaurate(III) ($\geq$99,9\%) & \\ 
				borane tert-butylamine(BBA, 97\%) &\\
				Tetralin (99\%)&\\
				Oleylamine ()&\\
				3-Mercaptopropionic acid (MPA,)&\\
				Kaliumhydroxid (KOH,99,9\%)&\\
				Silbernitrat&\\
				Natriumcitrat&\\
				Natriumborohydrid&\\
				Yttrium(III)chlorid Hexahydrat&\\
				Ytterbium(III)chlorid Hexahydrat&\\
				\bottomrule
			\end{tabular}
		\end{table}
	\subsection{Herstellung von Gelen}
		\subsubsection{Herstellung von Hydrogelen durch Yttrium und Ytterbium}
		\todo{Ref, Was wird geamcht?}
		
		
			\paragraph{Synthese von Gold-Nanopartikel in organischen Lösungsmitteln}
		
			\SI{200}{\milli\gram} Hydrogen tetrachloroaurate(III) werden gelöst in einer Lösung aus \SI{10}{\milli\liter} Tetralin und \SI{10}{\milli\liter} Oleylamine und in einem Eisbad für 20 Minuten gerührt.
			\SI{0,087}{\gram} BBA werden in \SI{1}{\milli\liter} Tetralin und \SI{1}{\milli\liter} Oleylamine gegeben und in einem Ultraschallbad gelöst und anschließend in gekühlte Lösung injiziert.
			Anschließen wird das Reaktionsgemisch für weitere \SI{2}{\hour} im Eisbad gerührt.
			Daraufhin wird die Lösung auf 2 \SI{50}{\milli\liter}-Zentrifugengefäße gleichmäßig aufgeteilt und zum Fällen der Partikel mit Aceton auf \SI{50}{\milli\liter} aufgefüllt und bei 8500~rpm für \SI{10}{\minute} zentrifugiert.
			Nach dem Zentrifugieren wurden die Partikel in \SI{3}{\milli\liter} je Zentrifugengefäß redispergiert.
			%Anschließend wurden die beiden Lösungen wieder zusammen in ein 25ml Schraubdeckelgläschen gegeben. Name: BG-AU-NP1. Verweis auf Liu2015
				
			\paragraph{Phasentransfer der Gold-Nanopartikel}
		
			Für den Phasentransfer werden je \SI{1}{\milli\liter} der Goldlösung in \SI{5}{\milli\liter} einer methanolischen \SI{0,1}{M}~KOH-Lösung gegeben und anschließend \SI{250}{\micro\liter} MPA dazugegeben und für \SI{4}{\hour} geschwenkt.
			Nach dem Abzentrifugieren (8500~rpm, 10 Minuten) wird der Rückstand in 4mL wässriger KOH-Lösung aufgenommen und dann \SI{1}{\milli\liter} Chloroform dazugegeben. 
			Da nach erneutem Zentrifugieren sich kein fester Rückstand gebildet hatte, wurden \SI{10}{\milli\liter} Aceton dazugegeben und das Zentrifugieren wiederholt.
				
			\paragraph{Gelierung der Gold-Nanopartikel mit Yttrium und Ytterbium}
				
			Es werden von den phasentransferierten Gold-Nanopartikeln, nach vorheriger Bestimmung der Goldkonzentration durch AAS-Messungen, Lösungen mit den Konzentrationen 0,625; 1,5 und \SI{2,5}{\gram\per\liter} hergestellt.
			Mit Yttrium(III)chlorid Hexahydrat bzw. Ytterbium(III)chlorid Hexahydrat werden je 3 Lösungen (1; 5; \SI{10}{mM}) hergestellt.
			Es werden immer \SI{200}{\micro\liter} Goldlösung und \SI{25}{\micro\liter} Yttrium- bzw. Ytterbium-Lösung dazugegeben zusammen in ein \SI{2}{\milli\liter}-Zentrifugengefäß gegeben und für \SI{24}{\hour} stehen gelassen.
			Dies wurde für jede Kombination aus Goldlösung und Y/Yb-Lösung, insgesammt 18, durchgeführt.
			Es bildet sich bei allen unten im Gefäß ein kleiner dunkler Klumpen.
			  
		\subsubsection{Herstellung von Hydrogelen durch Wasserstoffperoxid}
			Hier werden im ersten Schritt Gold- und Silbernanopartikel in wässriger Lösung synthetisiert und aufkonzentriert.
			Im zweiten Schritt wird dann die Gelierung ausgelöst. \autocite{Bigall2009} 
		 
			\paragraph{Synthese Goldnanopartikel in wässriger Phase mit Natriumcitrat}
			
			\SI{29}{\milli\liter} 0,2\%ige Goldchlorid-Lösung werden in \SI{500}{\milli\liter} dest. Wasser gegeben.
			Es werden \SI{11,6}{\milli\liter} einer 1\%igen Natriumcitratlösung dazugegeben und nach 30 Sekunden \SI{5,8}{\milli\liter}	einer eiskalten Lösung aus \SI{0,085}{\gram} Natriumborohydrid und \SI{0,5}{\gram} Natriumcitrat in \SI{50}{\milli\liter} dest. Wasser dazugegeben.
			Anschließend werden die Goldpartikel durch Zenrifugenfilter auf 10mL aufkonzentriert.
			
			\paragraph{Synthese Silbernanopartikel in wässriger Phase mit Natriumcitrat}
			
			\SI{12}{\milli\liter} 0,2\%ige Silbernitrat-Lösung werden in \SI{488}{\milli\liter} dest. Wasser gegeben und auf \SI{100}{\degreeCelsius} erhitzt.
			Anschließend werden \SI{11,6}{\milli\liter} einer 1\%igen Natriumcitratlösung dazugegeben.
			Nach 30 Sekunden werden  \SI{5,5}{\milli\liter}	einer eiskalten Lösung aus \SI{0,038}{\gram} Natriumborohydrid und \SI{0,5}{\gram} Natriumcitrat in \SI{50}{\milli\liter} dest. Wasser dazugegeben.
			Nach 2 Minuten wird alles im Dunkeln abgekühlt.
			Anschließend werden die Silberpartikel durch Zenrifugenfilter auf 10mL aufkonzentriert.
			
			\paragraph{Gelierungsprozess}
			    \begin{description}
			    \item[Gele aus Gold-Nanopartikeln:]
			    \SI{500}{\micro\liter} der Goldlösung werden in ein 2mL-Zentri"-fugen"-gefäß gegeben und anschließend \SI{40}{\micro\liter} einer 1\%igen Wasserstoffperoxid-Lösung gegeben und anschließend für X Tage dunkel gelagert.
			    \item[Gele aus Silber-Nanopartikeln:]
			    \SI{500}{\micro\liter} der Silberlösung werden in ein 2mL-Zentri"-fugen"-gefäß gegeben und anschließend \SI{200}{\micro\liter} einer 3\%igen Wasserstoffperoxid-Lösung gegeben und anschließend für X Tage dunkel gelagert.
			    \item[Bimetallisches Gel aus Gold/Silber-Nanopartikeln:]
			    \SI{500}{\micro\liter} der Goldlösung und 	\SI{500}{\micro\liter} der Silberlösung werden in ein 2mL-Zentri"-fugen"-gefäß gegeben und anschließend für X Tage dunkel gelagert.
			    \end{description}
			    
			
		\subsubsection{Ethanolischer Ansatz für direkte Synthese von Gelen}
			Im Gegensatz zu den vorherigen Ansätzen, sind hier Nanopartikelbildung und Gelierung keine zwei voneinander getrennte Syntheseschritte.
			Hier werden direkt aus der ethanolischen Metallsalzlösung die Gele gebildet. \autocite{Georgi2019}
			
			Unter Schutzatmosphäre werden \SI{39}{\milli\gram} Hydrogen tetrachloroaurate(III) in \SI{8}{\milli\liter} trockenem Ethanol gegeben.
			Eine zweite Lösung wird hergestellt aus \SI{11}{\milli\gram} NaBH$_4$ in \SI{6}{\milli\liter} trockenem Ethanol gegeben.
			Es werden 6 Proben präpariert, indem jeweils \SI{1,33}{\milli\liter} der Goldlösung in ein \SI{8}{\milli\liter}-Schraubdeckelglas gegeben wird und anschließend \SI{1}{\milli\liter} der zweiten Lösung dazugegeben wird.
			Die Lösung färbt sich sofort dunkel.
			Nach einer Stunde hat sich die Lösung klar gefärbt und es hatte sich entweder ein dunkel Bodensatz gebildet oder es schwamm ein Klumpen an der Oberfläche.
			Bei den Proben mit Bodensatz wurden die Gläser leicht schräg gehalten und gedreht, wodurch sich der Bodensatz zusammenklumpte und wie bei den anderen Proben, dieser Klumpen dann an der Oberfläche schwamm.  
			
	\subsection{Synthese von Kupfersulfid durch Single-Source-Precurser}
	    Für die Synthese von Kupfersulfid wird eine thermische Zersetzung eines Kupferkomplexes genutzt. \autocite{JenLaPlante2010}
		\subsubsection{Synthese des Single-Source-Precursors \ch{Cu[DDTC]2}}
	        Zur Herstellung des gewünschten Precursors (\ch{Cu[DDTC]2}) werden zwei Lösungen vorbereitet: 
    	    Einmal wurden \SI{0,2218}{\gram} \ch{CuCl2} in \SI{20}{\milli\liter} Wasser gelöst.
	        Für die zweite Lösung wurden \SI{0,7323}{\gram} \ch{Na[DDTC]* 3 H2O} in \SI{20}{\milli\liter} Wasser gelöst.
	        Anschließend werden beide Lösungen zusammengegeben, wobei sich direkt ein dunkler Niederschlag bildet.
	        Dieser wird abzentrifugiert und an Luft getrocknet.
	        Daraufhin wird es in heißem Chloroform rekristallisiert und erneut an Luft getrocknet.
		    Das entstandene \ch{Cu[DDTC]2} wird abgewogen und in TOP aufgenommen, sodass eine 0,11 molare Lösung erhalten wird.
		    
		\subsubsection{Herstellung Kupfersulfid}
		    Es werden \SI{100}{\micro\liter} der 0,11~M \ch{Cu[DDTC]2}-Lösung in ein \SI{8}{\milli\liter}-Schraubdeckelgläschen gegeben, dass in einem Sandbad auf \SI{290}{\degreeCelsius} unter Stickstoffatmosphäre erhitzt wird.
		    Dies wird solange erhitzt, bis das gesamte Lösemittel verdampft ist und ein schwarzer trockener Rückstand erkennbar ist.
		    Dieser wird in \SI{0,5}{\milli\liter} Toluol aufgenommen, anschließend \SI{1}{\milli\liter} Methanol dazugegeben und abzentrifugiert.
		    Der Rückstand wird in \SI{0,5}{\milli\liter} Toluol aufgenommen.
	
	\subsection{Synthese von CuS mit \ch{Cu[DDTC]2} und \ch{CuCl2}}
	    Da bekannt ist, dass Chloridionen einen starken Einfluss auf das Nukleationsverhalten bei Nanopartikeln haben kann\cite{Hinrichs2016}, wurde ein Teil des Single-Source-Precursors durch \ch{CuCl2} substituiert.
	    
	    
	\subsection{Synthese aus Goldnanopartikeln und Kupfer-diethyldithiocarbonat}
	
		Bei dieser Synthese wurde die CuS-Synthese, wie oben beschrieben, durchgeführt, mit einem Anteil an Gold-Nanopartikeln, mit der Absicht, dass die CuS-Bildung um die Nanopartikel herum stattfindet.	
	    Es werden \SI{100}{\micro\liter} der 0,11~M \ch{Cu[DDTC]2}-Lösung und verschiedene Mengen (\cref{tab:Au_Cu_Ratio} der Goldnanopartikel in Hexan in ein \SI{8}{\milli\liter}-Schraubdeckelgläschen zusammengegeben und in einem Sandbad unter Stickstoffatmosphäre auf \SI{290}{\degreeCelsius} erhitzt, bis das Lösemittel komplett verdampft ist.
	
    \begin{table}[H]
		\centering
		\caption{Die verschiedenen genutzten Verhältnisse aus Goldnanopartikeln und \ch{Cu[DDTC]2}.}
		\label{tab:Au_Cu_Ratio}
		\begin{tabular}{ll}
            \toprule
            V(Au)/\si{\micro\liter}&V(\ch{Cu[DDTC]2})/\si{\micro\liter}\\
            \midrule
            10&100\\
            20&100\\
            30&100\\
            40&100\\
            50&100\\
            100&100\\
            \bottomrule
        \end{tabular}
    \end{table}
	