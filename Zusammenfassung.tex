% !TEX root = BIGALL-Master.tex
\section{Zusammenfassung}

Ziel der Arbeit war es eine Methode zu entwickeln an vorhandenen Nanopartikelgelen andere Materialien heranzuwachsen.
Die hierfür gewählten Materialien waren metallische Nanopartikelgele, an denen Halbleitermaterialien herangewachsen wurden.

Da bei der Arbeit mit Gelen zu hohe Temperaturen zu vermeiden waren und gleichzeitig eine Synthesemethode gewählt werden musste, bei der die Edukte nicht sofort bei Zugabe reagierten, sondern genug Zeit haben in die inneren Poren der Gele vorzudringen, wurde als grundlegende Methode hier die thermische Zersetzung von Single-Source-Precursoren verwendet.

Als Single-Source-Precursoren wurden Diethyldithiocarbamat-Komplexe verwendet, da sich diese schon bei Temperaturen von unter \SI{300}{\degreeCelsius} zu Metallsulfiden zersetzen.
Aus disem Grund wurden die Komplexe \ch{Cu[DDTC]2}, \ch{Cd[DDTC]2} und \ch{Zn[DDTC]2} hergestellt und ihre Bildung von Nanopartikeln an Gelen untersucht.

Bei der Synthese der Gele wurden verschiedene bekannte Methoden verwendet.
Es wurde zum einen ein klassischen Verfahren zur Herstellung von Gelen aus citratstabilisierten Nanopartikel durch Ligandenoxidation durch \ch{H2O2} verwendet,
zum anderen ein neuerer Ansatz bei dem die Nanopartikel durch Zugabe von Yttrium- und Ytterbiumchlorid zur Gelierung gebracht wurden.
Des Weiteren wurde ein neueres Verfahren verwendet, bei dem Nanopartikelbildung und Gelbildung in einem Schritt zusammengefasst sind.
Vor allem Reaktionen mit den zuletzt genannten Gelen wurden in dieser Arbeit untersucht.

Es zeigte sich, dass sich durch die thermische Zersetzung der Single-Source-Precursoren Anlagerungen an den Gelen bildeten.
Je nach verwendetem Precursor variierten diese jedoch von fast kompletter Ummantelung des Gels bei \ch{Cd[DDTC]2} bis zu keiner Anlagerung bei \ch{Zn[DDTC]2}, was durch die Zugabe von Zinkchlorid allerdings verbessert werden konnte.
Es konnte sowohl bei den Gelen aus ethanolischen Ansatz als auch aus den citratstabilisierten Nanopartikeln Anlagerungen beobachtet werden.

Insgesamt konnte hier also mit dem Prinzip der Zersetzung von Single-Source-Precursoren eine Methode entwickelt werden, mit der das nachträgliche Auftragen von Partikeln an Gelen ermöglicht wird.

Weiterführend könnte man Tests mit anderen Metallen in Diethyldithiocarbamat-Komplexen durchführen oder andere Komplexe wie Dimethyldithiocarbamat- oder Hexylmethyldithiocarbamat-Komplexe verwenden.
Des Weiteren kann dieser Prozess auch mit anders beschaffenen Gelen verwendet werden.


