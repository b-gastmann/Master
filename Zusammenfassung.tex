% !TEX root = BIGALL-Master.tex
\section{Zusammenfassung}

Ziel der Arbeit war es eine Methode zu entwickeln an vorhandenen Nanopartikelgelen andere Materialien heranzuwachsen.
Die hierfür gewählten Materialien waren metallische Nanopartikelgele an denen Halbleitermaterialien herangewachsen wurde.

Da bei der Arbeit mit Gelen zu hohe Temperaturen zu vermeiden waren und gleichzeitig eine Synthesemethode gewählt werden musste, bei der die Edukte nicht sofort bei Zugabe reagierten, sondern genug Zeit haben in die inneren Poren der Gele vorzudringen wurde als grundlegende Methode hier die thermische Zersetzung von Single-Source-Precursoren verwendet.

Als Single-Source-Precursoren wurden   verwendet, da sich diese schon bei Temperaturen von unter \SI{300}{\degreeCelsius} zu Metallsulfiden zersetzen.
Die Komplexe wurden \ch{Cu[DDTC]2}, \ch{Cd[DDTC]2} und \ch{Zn[DDTC]2} hergestellt und ihre Bildung von Nanopartikeln an Gelen untersucht.

Bei der Synthese der Gele wurden verschiedene bekannte Methoden verwendet.
Es wurde zum einen ein klassischen Verfahren zur Herstellung von Gelen aus citratstabilisierten Nanopartikel durch Ligandenoxidation durch \ch{H2O2} verwendet,
zum anderen ein neuerer Ansatz bei dem die Nanopartikel durch Zugabe von Yttrium und Ytterbium zur Gelierung gebracht werden.
Des weiteren wurde ein neueres Verfahren verwendet, bei dem Nanopartikelbildung und Gelbildung in einem Schritt zusammengefasst sind.
Vor allem Reaktionen mit den zuletzt genannten Gelen wurden in dieser Arbeit untersucht.

Es zeigte sich, dass sich durch die thermische Zersetzung der Single-Source-Precursoren Anlagerungen an den Gelen bildeten.
Je nach verwendetem Precursor variierten diese jedoch von fast kompletter Ummantelung des Gels bei \ch{Cd[DDTC]2} bis zu keiner Anlagerung bei \ch{Zn[DDTC]2}, was durch die Zugabe von Zinkchlorid allerdings verbessert werden konnte.
Es konnte sowohl bei den Gelen aus ethanolischen Ansatz als auch aus den citratstabilisierten Nanopartikeln Anlagerungen beobachtet werden.

Insgesamt konnte hier also mit dem Prinzip der Zersetzung von Single-Source-Precursoren eine Methode entwickelt werden, mit der das nachträgliche Auftragen von Partikeln an Gelen ermöglicht wird.

Weiterführend könnte man Tests mit anderen Metallen in Diethyldithiocarbamat-Komplexen durchführen oder andere Komplexe wie Dimethyldithiocarbamat- oder Hexylmethyldithiocarbamat-Komplexe verwenden.
Des weiteren kann dieser Prozess auch mit anders beschaffenen Gelen verwendet werden.


\chapter*{Danksagung}

Die vorliegende Masterarbeit entstand im Arbeitskreis von Prof. Dr. Nadja Bigall am Institut für Physikalische Chemie und Elektrochemie der Leibniz Universität Hannover. Ich möchte mich an dieser Stelle bei allen bedanken, die mich während dieser Arbeit sowohl fachlich als auch moralisch unterstützt haben. Mein besonderer Dank gilt
dabei vor allem:

\begin{itemize}
	\item \textbf{Frau Prof. Dr. Nadja Bigall} für die Möglichkeit diese Arbeit in ihrem Arbeitskreis zu verfassen und ihre stetige Bereitschaft für produktive und interessante Hinweise.
	\item \textbf{Herrn Prof. Dr. Peter Behrens} dafür, dass er als Zweitprüfer zur Verfüngung stand.
	\item \textbf{Pascal Rusch} für seine stets hilfsbereite Art und die TEM-Messungen.
	\item Dem gesamten Arbeitskreis Bigall für das tolle Arbeitsklima.
	\item Meiner Familie und meinen Freunden die immer für mich da sind.
	
\end{itemize}

\pagebreak

\chapter*{Eidesstattliche Erklärung}

Hiermit versichere ich, die vorliegende Masterarbeit selbstständig, ohne
fremde Hilfe und ohne Benutzung anderer als der von mir angegebenen Quellen
angefertigt zu haben. Alle aus fremden Quellen direkt oder indirekt übernommenen
Gedanken sind als solche gekennzeichnet. Die Arbeit wurde noch
keiner Prüfungsbehörde in gleicher oder ähnlicher Form vorgelegt.

\vspace{5cm}

\noindent Hannover, den 31. März 2020
\vspace{2cm}
\hrule
\vspace{5mm}
\noindent Björn Gastmann